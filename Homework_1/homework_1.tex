\documentclass[12pt]{article}

\usepackage[utf8]{inputenc}
\usepackage[english, russian]{babel}
\usepackage{amsthm}

\title{Домашнее задание 1. \\ Введение в алгебру логики}
\author{Балдин Виктор Б01-303}

\begin{document}
    \maketitle

    \begin{enumerate}
        \item Доказать:
            \[1\oplus x_1 \oplus x_2 = (x_1 \to x_2) \wedge (x_2 \to x_1)\]
        \textbf{Доказательство: }
        \[1 \oplus x_1 \oplus x2 = \lnot(x1 \oplus x2) = x_1\leftrightarrow
        x_2\]
        \[ (x_1 \to x_2) \wedge (x_2 \to x_1) = (\lnot x_1 
        \vee x_2) \wedge (\lnot x_2 \vee x_1) = \lnot 
        x_1 \wedge \lnot x_2 \vee x_1 \wedge x_2 = x_1 \leftrightarrow x_2\]
        \qed
        
        \item
        \subitem (a) \( x \wedge (y \to z) = (x \wedge y) \to
        (x \wedge z) \)
        \subitem (b) \( x \oplus (y \leftrightarrow z) = (x \oplus y) 
        \leftrightarrow (x \oplus z) \)

        \textbf{Решение:}
        \subitem (a) \( x \wedge (y \to z) = x \wedge (\lnot y \vee z)
        = x \wedge \lnot y \vee x \wedge z \neq \lnot(x \wedge y) \vee x \wedge
        z = (x \wedge y) \to (x \wedge z)\) -- \textit{ложно}
        \subitem (b)
        \( f = x \oplus (y \leftrightarrow z) = 10010110 \\
           g = (x \oplus y) \leftrightarrow (x \oplus z) = 10011001 \\
           f \neq g
        \) -- \textit{ложно}
        
        \item 
        \subitem (a) \( 1 \to 0 \neq 0 \to 1\) нет
        \subitem (b) \( x \to (y \to z) = \lnot x \vee (\lnot y \vee z)
        = \lnot x \vee \lnot y \vee z = \lnot (x \wedge y) \vee z \neq \\
        (x \to y) \to z = x \wedge \lnot y \vee z \)
        нет

        \item 
        \subitem (a) \( y \)
        \subitem (b) \( z \)
        \subitem (c) \( k(x_1, x_2, x_3) =
        (x_1 \to (x_1 \vee x_2)) \to x_3 = (\lnot x_1 \vee x_1 \vee x_2)
        \to x_3 = (1 \vee x_2) \to x_3 = 1 \to x_3 = x_3\) -- фиктивные
        переменные $x_1$, $x_2$

        \item Данное выражение можно угадать, основываясь на 
        соображениях симметрии:
        \[ MAJ(x_1, x_2, x_3) = (x_1 \vee x_2) \wedge (x_2 \vee x_3)
        \wedge (x_1 \vee x_3) = 00010111\]

        \item Доказать: 
        \[ \bigvee (x_i \oplus x_j) = (x_1 \vee \dots \vee 
        x_n) \wedge (\lnot x_1 \vee \dots \vee \lnot x_n ) \]
        \textbf{Доказательство:} \\
        \textit{Случай 1.} Пусть \( \exists i, j: x_i \neq x_j \). Тогда 
        \( x_i \oplus x_j = 1 \Rightarrow \bigvee (x_i \oplus x_j) = 1 \);
        \( x_1 \vee \dots \vee x_n = 1\); \( \lnot x_1 \vee \dots
        \vee \lnot x_n = 1 \), т. к. для этого случая
        \( \exists i, j: x_i = 1, x_j = 0 \)\\
        \textit{Случай 2.} \( \forall i, j: x_i = x_j \Rightarrow 
        \bigvee (x_i \oplus x_j) = 0; (x_1 \vee \dots
        \vee x_n) \wedge (\lnot x_1 \vee \dots \lnot x_n) = 
        x_1 \wedge \lnot x_1 = 0 \)
        \qed 

        \item Например \( f(x, y) = \lnot (x \wedge y) \).
        Выражения для операций через $f$ (достаточно всего 2-х, для
        отрицания и дизъюнкции, например):
        \[ f(x, x) = \lnot (x \wedge x) = \lnot x \]
        \[ f(f(x, x), f(y, y)) = \lnot (\lnot x \wedge \lnot y) =
        x \vee y \]
    \end{enumerate}
\end{document}